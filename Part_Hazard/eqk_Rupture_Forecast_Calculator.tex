The Earthquake Rupture Forecast calculator creates a list of all the possible 
ruptures


%
%  . . . . . . . . . . . . . . . . . . . . . . . . . . . . . . . . . . . . . . . 
\subsubsection{The Poisson model}
The Poisson distribution gives the probability of occurrence of $n$ events in a 
time interval $t$, provided a value of the occurrence rate $\lambda$: 
\begin{equation}
P(N=n|t,\lambda)=\frac{(\lambda t)^{n}\exp(-\lambda t)}{n!}
\end{equation}
%
%  . . . . . . . . . . . . . . . . . . . . . . . . . . . . . . . . . . . . . . . 
\subsubsection{Brownian Time Passage (BTP) model}
The Brownian Time Passage model was originally proposed by \cite{matthews2002}. 
Not extensively used in PSHA; the Japan J-SHIS is one representative model 
containing this temporal occurrence model.
%
%  - - - - - - - - - - - - - - - - - - - - - - - - - - - - - - - - - - - - - - -
\subsection{Frequency-magnitude distribution models}
The frequency-magnitude distribution describes the density of earthquakes of a 
given magnitude in a given time interval.
%
%  . . . . . . . . . . . . . . . . . . . . . . . . . . . . . . . . . . . . . . . 
\subsubsection{Gutenberg-Richter distribution}
Truncated Gutenberg-Richter distribution.

%
% ------------------------------------------------------------------------------
\section{ERF creation in case of Area sources}
Area sources (see also Section \ref{hazard:seismic_source_types:areaSources} 
at page \pageref{hazard:seismic_source_types:areaSources}) 

%
% ------------------------------------------------------------------------------
\section{ERF creation in case of Grid sources}

%
% ------------------------------------------------------------------------------
\section{ERF creation in case of Fault sources}

%
%  - - - - - - - - - - - - - - - - - - - - - - - - - - - - - - - - - - - - - - - 
\subsection{Fault sources with simple geometry}

%
%  - - - - - - - - - - - - - - - - - - - - - - - - - - - - - - - - - - - - - - - 
\subsection{Fault sources with complex geometry}
