OpenQuake computes classical PSHA 
\citep{cornell1968,mcguire2004} following the methodology presented by 
\citet{field2003}. This methodology has the distinctive property of performing
the entire calculation using probabilities, as originally proposed by 
\citet{chiang1984}, instead of working with occurrence rates like in most 
of the commonest PSHA codes \citep[see for instance][]{bender1987}. 
%
The OpenSHA methodology has also the clear advantage of decoupling the creation of the probabilistic seismicity occurrence model (in the OpenSHA terminology this is defined as the Earthquake Rupture Forecast) from the assumption of a Poissonian temporal occurrence model. 

As demonstrated by \citet{field2003} - under the the assumption that the contribution to hazard coming from multiple occurrences occurring on a given  seismic source is negligible i.e. the probability that a source will generate two or more occurrences within the time span fixed for the analysis is equal to zero - the calculation of hazard following this methodology is completely consistent with the most classical procedure.
% 
\citet{pagani2007} proved that this assumption holds at least for some prototypal situations.

Two are the main steps composing the OpenSHA seismic hazard calculation 
procedure:
%
\begin{itemize}
\item Creation of the probabilistic seismicity occurrence model, i.e. a discrete 
distribution giving the probability of occurrence - in a given time span - of each possible rupture occurring on one of the seismic sources defined in the PSHA input model. This step of the procedure was already discussed in 
Chapter \ref{chap:erf} at page \pageref{chap:erf}.
%
\item Calculation of hazard at the site by combining the probabilistic 
seismicity occurrence model with a ground motion prediction equation (also Intensity Measure Relationship in the OpenSHA jargon). This second step will be the topic of the current chapter.
\end{itemize}
%
% --------------------------------------------------------------------------------
\section{Hazard calculation: traditional formulation in terms of probabilities}
In the simplest case, the classical PSHA calculation kernel takes as an input: 
%
\begin{itemize}
\item An Earthquake Rupture Forecast (ERF - also called Probabilistic Seismicity Occurrence Model). An ERF is a list of all the possible ruptures occurring on all the seismic sources included in a Source Model. Each rupture $Rup$ is associated with a probability of occurrence $P(Rup|t)$ referred to the time span $t$ fixed for the analysis. 
\item A Ground Motion Prediction Equation (GMPE). A GMPE is an equation that - given some fundamental parameters characterizing the source, the propagation
path and the site (in the simplest case magnitude, distance and V$_\text{S,30}$) - computes the value $GM$ of a (scalar) parameter describing ground motion intensity. $gm$ is always accompanied by a standard deviation value $\sigma_{gm,T}$ specifying the aleatoric variability associated to $gm$.
\end{itemize}
%
Following \cite{field2003}, the exceedance of a value probabilities :
%
\begin{equation}
P(U\geq u)= 
	1-\prod\limits_{i=1}^{l} 
	\left[\sum\limits_{s=0}^{+\infty}
	\left(P(S=s) 
	\left(
		1-\sum\limits_{j=0}^{j(i)}\sum\limits_{s=0}^{K(i,j)} 
		P(m_{i,j}) 
		P(R_{i,j,k}|m_{i,j}) P(U\geq u|m_{i,j},R_{i,j,k})
	\right)
	\right)^{s}
	\right] 
\end{equation}
where $l$ i the number of sources 

If the probability of multiple occurrences is assumed to be negligible the 
probability to get an exceedance of $x$ in a given time span corresponds to:
%
\begin{equation}
P(X\geq x)=1-\prod\limits_{i=1}^{l} 
	\left( 
		1-\sum\limits_{n=1}^{N(i)}P(Rup_{i,n})P(X\geq x|Rup_{i,n})
	\right)
\end{equation}

%  - - - - - - - - - - - - - - - - - - - - - - - - - - - - - - - - - - - - - - - -
\subsection{Example}
In case of two punctual sources each one generating a single rupture, the 
probability of exceedance of ground motion $u$ in a given site corresponds to:
%
\begin{eqnarray}
P(U\geq u)=
	1-
	\biggl(& 
		\bigl[ 1-P(Rup_{i,1})P(U\geq u|Rup_{i,1}) \bigr] \nonumber \\
		& \bigl[ 1-P(Rup_{i,2})P(U\geq u|Rup_{i,2}) \bigr]
	\,\biggr)
\end{eqnarray}